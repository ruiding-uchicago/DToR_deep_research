\section{Supplemental Results: Experimental Setup}

\subsection{Nanomaterials/Devices Topic Suite}\label{app:topics}

% --- 1st PFAS 2D FET Probes ---
\begin{tcolorbox}[title={PFAS\_2D\_FET\_Probes}, breakable]
\begin{Verbatim}[breaklines=true, breakanywhere=true, fontsize=\small]
Which two-dimensional (2D) nanomaterials or molecular probes—including novel, underexplored candidates—offer the greatest potential to achieve highly sensitive and selective detection of per- and polyfluoroalkyl substances (PFAS), or more broadly chemical and biological analytes, when integrated into a field-effect transistor (FET)–based sensor platform?
\end{Verbatim}
\end{tcolorbox}

% --- 2nd LIB Advanced Binders ---
\begin{tcolorbox}[title={LIB\_Advanced\_Binders}, breakable]
\begin{Verbatim}[breaklines=true, breakanywhere=true, fontsize=\small]
Which advanced binder technologies are being developed to improve the performance and longevity of lithium-ion batteries?
\end{Verbatim}
\end{tcolorbox}

% --- 3rd S. aureus FET (commercial probes) ---
\begin{tcolorbox}[title={S\_aureus\_FET\_Commercial\_Probes}, breakable]
\begin{Verbatim}[breaklines=true, breakanywhere=true, fontsize=\small]
How can emerging probe chemistries that are commercially available, cost-effective, and exhibit minimal batch-to-batch variability be used to develop a novel FET-based biosensor for detecting Staphylococcus aureus? 
\end{Verbatim}
\end{tcolorbox}

% --- 4th Resource Recovery Electrocatalysts ---
\begin{tcolorbox}[title={Wastewater\_Resource\_Recovery\_Cats}, breakable]
\begin{Verbatim}[breaklines=true, breakanywhere=true, fontsize=\small]
Which nanostructured electrocatalyst materials demonstrate the highest selectivity and efficiency for electrochemical detection and recovery of critical resources (e.g., Li⁺, PO₄³⁻, NH₄⁺) from complex wastewater matrices, and what key performance metrics—such as selectivity, sensitivity, recovery rate, energy consumption, and operational stability—distinguish them?
\end{Verbatim}
\end{tcolorbox}

% --- 5th OER in real wastewater ---
\begin{tcolorbox}[title={OER\_Wastewater\_Resistant\_Cats}, breakable]
\begin{Verbatim}[breaklines=true, breakanywhere=true, fontsize=\small]
Identify the top nanostructured electrocatalyst materials for driving the oxygen evolution reaction (OER) in complex wastewater matrices—such as high‐chloride, high‐organic‐load, or multi‐ion streams—,  for each the material class/composition, comprehensively consider key performance metrics (overpotential at 10 mA cm⁻², Faradaic efficiency, stability), and the surface‐engineering strategies that confer corrosion resistance and sustained activity in real effluent conditions.
\end{Verbatim}
\end{tcolorbox}

% --- 6th Photothermal wastewater evaporation ---
\begin{tcolorbox}[title={Solar\_Evap\_Wastewater\_Photothermal}, breakable]
\begin{Verbatim}[breaklines=true, breakanywhere=true, fontsize=\small]
What photothermal materials and system designs are most effective for solar-driven water evaporation in complex wastewater matrices—such as high-organic-load or multi-ion streams—and how do they compare in terms of material composition, solar-to-vapor conversion efficiency under one-sun illumination, evaporation rate, fouling resistance, and integrated resource-recovery functionalities?
\end{Verbatim}
\end{tcolorbox}

% --- 7th Nutrient sensor anti-interference ---
\begin{tcolorbox}[title={Nutrient\_Sensors\_Anti\_Interference}, breakable]
\begin{Verbatim}[breaklines=true, breakanywhere=true, fontsize=\small]
Which sensor probe materials/chemicals and designs offer the best performance in minimizing interference factors—such as competing ions, dissolved organic matter, pH fluctuations, and temperature variations—for accurate and selective detection of nutrients (e.g., nitrate, phosphate, ammonium) in complex water matrices, and what mitigation strategies do they employ?
\end{Verbatim}
\end{tcolorbox}

% --- 8th CO2 sensing with 2D materials ---
\begin{tcolorbox}[title={CO2\_Sensing\_2D\_Materials}, breakable]
\begin{Verbatim}[breaklines=true, breakanywhere=true, fontsize=\small]
Which two-dimensional materials—such as graphene derivatives, transition metal dichalcogenides, or MXenes—offer the highest CO₂ sensing performance in complex gas or aqueous environments, and how do they compare in terms of detection limit (ppm), selectivity against common interferents (e.g., O₂, H₂O), response/recovery time, and long-term stability, including any functionalization or structural modifications that enhance these metrics? Think of novel candidates.
\end{Verbatim}
\end{tcolorbox}

% --- 9th Printed FET array variability ---
\begin{tcolorbox}[title={Printed\_FET\_Array\_Process\_Window}, breakable]
\begin{Verbatim}[breaklines=true, breakanywhere=true, fontsize=\small]
Which printed‐electronics fabrication parameters and post‐processing strategies—such as ink viscosity, printing speed and resolution, substrate surface energy, annealing temperature profiles, and in‐line calibration methods—have been shown to minimize device‐to‐device variability in FET sensor arrays, and what specific process windows achieve low variation in threshold voltage and field‐effect mobility?
\end{Verbatim}
\end{tcolorbox}

% --- 10th 2D Synaptic Transistors (practical) ---
\begin{tcolorbox}[title={2D\_Synaptic\_Transistors\_Practical}, breakable]
\begin{Verbatim}[breaklines=true, breakanywhere=true, fontsize=\small]
Which two-dimensional material platforms (e.g., MoS₂, WSe₂, black phosphorus, h-BN), device architectures (e.g., floating-gate, ionic-gated, dual-gate), and fabrication protocols (e.g., channel thickness control, dielectric engineering, contact metallurgy) have been shown to optimize synaptic transistor performance—specifically in terms of energy per event, weight-update linearity, retention time, and cycling endurance—for neuromorphic sensing applications? Think of most practical and promising candidates.
\end{Verbatim}
\end{tcolorbox}

% --- 11th Microplastics sensing with 2D platforms ---
\begin{tcolorbox}[title={Microplastics\_Sensing\_2D}, breakable]
\begin{Verbatim}[breaklines=true, breakanywhere=true, fontsize=\small]
Which two-dimensional material platforms (e.g., graphene derivatives, transition metal dichalcogenides, MXenes), molecular recognition elements (e.g., molecularly imprinted polymers, aptamers, peptide receptors), and device integration strategies (e.g., FET, electrochemical impedance, photonic transduction) have demonstrated the highest sensitivity, selectivity against organic matter and ionic interferents, and rapid response times for detecting micro- and nanoplastic particles in complex water matrices? Think of novel candidates.
\end{Verbatim}
\end{tcolorbox}

% --- 12th Antibiotics sensing (2D + probes) ---
\begin{tcolorbox}[title={Antibiotics\_Sensing\_2D\_Probes}, breakable]
\begin{Verbatim}[breaklines=true, breakanywhere=true, fontsize=\small]
Which commercially available, cost-effective probe chemistries (e.g., thiolated DNA aptamers, antibody mimetics, conductive MIPs) combined with two-dimensional nanomaterial transducer platforms (e.g., WS₂ FETs, graphene field-effect sensors, nanotube-extended gate FETs) deliver the lowest detection limits, minimal batch variability, and robust performance for sensing trace levels of pharmaceutical antibiotics in diverse aqueous waterbody.
\end{Verbatim}
\end{tcolorbox}

% --- 13th Li over Na selective membranes ---
\begin{tcolorbox}[title={Li\_over\_Na\_Selectivity\_Membranes}, breakable]
\begin{Verbatim}[breaklines=true, breakanywhere=true, fontsize=\small]
What material or membrane exhibits the highest selectivity for Li⁺ over Na⁺ in aqueous systems, given their nearly identical hydrated ionic radii and solvation environments? Beyond crown-ether–functionalized polymers, what novel or unexpected materials—such as bioinspired ultrahigh-selectivity membranes or covalent organic framework nanochannels—might provide breakthrough Li⁺ discrimination over Na⁺? Find novel and promising candidates.
\end{Verbatim}
\end{tcolorbox}

% --- 14th PFAS degradation electrodes ---
\begin{tcolorbox}[title={PFAS\_Electro\_Degradation}, breakable]
\begin{Verbatim}[breaklines=true, breakanywhere=true, fontsize=\small]
Which novel electrode materials can achieve efficient PFAS degradation under ambient aqueous electrochemical conditions, delivering both high mineralization and defluorination rates? What intrinsic properties—such as PFAS adsorption affinity, reactive oxygen species generation capacity, and C–F bond activation energy—should be optimized to guide their discovery?
\end{Verbatim}
\end{tcolorbox}

% --- 15th Li–Co–Ni separation membranes ---
\begin{tcolorbox}[title={LiCoNi\_Selective\_Membranes}, breakable]
\begin{Verbatim}[breaklines=true, breakanywhere=true, fontsize=\small]
Which membrane materials can effectively separate Li⁺, Co²⁺, and Ni²⁺ ions from aqueous solutions by leveraging selective transport properties—such as tailored pore sizes, specific surface functionalizations, and charge affinities—and what membrane design principles optimize both selectivity and permeability? Think of potential novel, effective, and practical candidates.
\end{Verbatim}
\end{tcolorbox}

% --- 16th In-situ LC-TEM for MoS2 FET sensing ---
\begin{tcolorbox}[title={InSitu\_Liquid\_TEM\_MoS2\_Sensing}, breakable]
\begin{Verbatim}[breaklines=true, breakanywhere=true, fontsize=\small]
How can in situ liquid cell TEM be employed to directly visualize the real-time adsorption and structural changes of 2D MoS₂ nanosheets used in aqueous FET sensors during analyte binding? What fluid cell configurations and electron dose parameters are necessary to preserve native water–material interfaces while capturing high-resolution sensing events without beam-induced artifacts?
\end{Verbatim}
\end{tcolorbox}

% --- 17th CO2RR catalyst leaders + emerging ---
\begin{tcolorbox}[title={CO2RR\_SOTA\_and\_Emerging\_Cats}, breakable]
\begin{Verbatim}[breaklines=true, breakanywhere=true, fontsize=\small]
Given the current landscape of CO₂ electroreduction, which state-of-the-art catalyst platforms—such as oxide-derived copper, single-atom catalysts on nitrogen-doped carbon supports, or metal–organic framework-derived materials—demonstrate the highest activity and selectivity? Moreover, what emerging catalytic systems or novel heterostructures beyond these examples could feasibly outperform today’s leading electrocatalysts in terms of faradaic efficiency and stability?
\end{Verbatim}
\end{tcolorbox}

% --- 18th Unconventional PV platforms ---
\begin{tcolorbox}[title={NextGen\_Solar\_Unconventional}, breakable]
\begin{Verbatim}[breaklines=true, breakanywhere=true, fontsize=\small]
Perovskite–silicon tandem cells, organic photovoltaics, and quantum-dot solar cells currently represent the forefront of next-generation solar technologies. Beyond these established platforms, which unconventional material classes or innovative device architectures—such as chalcogenide perovskites, 2D semiconductor heterostructures, or ferroelectric photovoltaic systems—offer the most unexpected promise for achieving breakthroughs in efficiency, stability, and scalability?
\end{Verbatim}
\end{tcolorbox}

% --- 19th Ambient-pressure diamond growth ---
\begin{tcolorbox}[title={Ambient\_Pressure\_Diamond\_Growth}, breakable]
\begin{Verbatim}[breaklines=true, breakanywhere=true, fontsize=\small]
How can ambient-pressure diamond synthesis using a Ga–Ni–Fe–Si liquid-metal alloy at 1 atm and ~1025 °C be adapted to produce larger-area diamond films or oriented single crystals, and what key mechanistic steps govern nucleation and growth kinetics under these mild-pressure conditions?
\end{Verbatim}
\end{tcolorbox}

% --- 20th Diamane property tuning ---
\begin{tcolorbox}[title={Diamane\_Bandgap\_Functionalization}, breakable]
\begin{Verbatim}[breaklines=true, breakanywhere=true, fontsize=\small]
What chemical functionalizations or reaction pathways are most promising for tuning the electronic bandgap and mechanical stability of diamane-like 2D diamond films formed via sp²-to-sp³ conversion, and how might those modifications influence their integration into nanoelectronic devices?
\end{Verbatim}
\end{tcolorbox}

% --- 21st LIB fluorinated ether + anion receptors ---
\begin{tcolorbox}[title={LIB\_Fluoroether\_Anion\_Receptor}, breakable]
\begin{Verbatim}[breaklines=true, breakanywhere=true, fontsize=\small]
Search for novel fluorinated ether–based electrolyte candidates that incorporate covalently bound anion-receptor motifs (e.g. boron- or phosphorus-centered groups) to deliver oxidative stability beyond 5.6 V, Li⁺ transference numbers above 0.8, and minimal interfacial impedance on lithium-metal anodes. 
\end{Verbatim}
\end{tcolorbox}

% --- 22nd Mixed-material multi-sensor arrays (e-tongue) ---
\begin{tcolorbox}[title={ML\_e\_Tongue\_for\_Water}, breakable]
\begin{Verbatim}[breaklines=true, breakanywhere=true, fontsize=\small]
Which mixed-material (e.g., graphene/2D) multi-sensor arrays with on-chip QA/QC and pattern-recognition deliver robust mixture fingerprinting (metals + microbes + organics) in flowing water, with automated drift/fault handling for field deployment?
\end{Verbatim}
\end{tcolorbox}

% --- 23rd Batteryless/stretchable biosensors ---
\begin{tcolorbox}[title={Batteryless\_Stretchable\_Biosensors}, breakable]
\begin{Verbatim}[breaklines=true, breakanywhere=true, fontsize=\small]
Which tribo/piezo-powered, skin-conformal platforms (CNT/graphene networks, ionic gels) sustain continuous metabolite/ion monitoring without batteries—what power densities and duty-cycled readouts are practical for week-scale operation?
\end{Verbatim}
\end{tcolorbox}

% --- 24th Mixed-dimensional low-noise transducers ---
\begin{tcolorbox}[title={Mixed\_Dimensional\_Low\_Noise\_Transducers}, breakable]
\begin{Verbatim}[breaklines=true, breakanywhere=true, fontsize=\small]
Which 2D-0D/2D-1D/2D-3D vdW stacks (e.g., MoS₂–QD, CNT–graphene) best combine strong capture/absorption with high-mobility transport to push FET/photodetector responsivity and baseline stability in real media?
\end{Verbatim}
\end{tcolorbox}

% --- 25th Electro/photo-gated smart membranes ---
\begin{tcolorbox}[title={Electro\_Photo\_Gated\_Smart\_Membranes}, breakable]
\begin{Verbatim}[breaklines=true, breakanywhere=true, fontsize=\small]
Which electrically gated or photo-responsive nanochannels (MXene/COF/polymer hybrids) deliver on-demand selectivity and anti-fouling for ion separations and resource recovery in wastewater, and what gating “on/off” ratios and cycling lifetimes are state-of-the-art?
\end{Verbatim}
\end{tcolorbox}

% --- 26th Transient/biodegradable sensor nodes ---
\begin{tcolorbox}[title={Transient\_Biodegradable\_Sensor\_Nodes}, breakable]
\begin{Verbatim}[breaklines=true, breakanywhere=true, fontsize=\small]
Which dissolvable conductors, dielectrics, and encapsulants enable time-programmed water/soil sensors that operate stably for weeks then resorb without e-waste—what degradation kinetics and wireless/data-retention strategies are most practical?
\end{Verbatim}
\end{tcolorbox}

% --- 27th Catalytic mineralization of microplastics ---
\begin{tcolorbox}[title={Catalytic\_Mineralization\_of\_Microplastics}, breakable]
\begin{Verbatim}[breaklines=true, breakanywhere=true, fontsize=\small]
Which ROS-generating photocatalysts and micromotor architectures achieve verifiable depolymerization/mineralization of common MPs (PE, PP, PET) in real waters—what mineralization fractions, spectra, and by-product toxicology benchmarks define “true cleanup”?  
\end{Verbatim}
\end{tcolorbox}
